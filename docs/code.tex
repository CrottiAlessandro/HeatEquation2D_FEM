In this Homework we have to solve the following Heat Equation \eqref{eq: Heat_equation} with the Finite Element Method (FEM). The domain is $\Omega \subset \mathbb{R}^2$ and it can be seen in Fig, \ref{fig: domain}.

\begin{equation}\label{eq:Heat_equation}
\begin{aligned}
    \text{Given} \quad &f(\mathbf{x},t) : \Omega \times \mathbb{T} \longrightarrow \mathbb{R}, \ d_1(\mathbf{x},t):\Gamma_{D,1} \times \mathbb{T} \longrightarrow \mathbb{R},\\
    &d_2(\mathbf{x},t):\Gamma_{D,2} \times \mathbb{T} \longrightarrow \mathbb{R}, \ q(\mathbf{x},t):\Gamma_{N} \times \mathbb{T} \longrightarrow \mathbb{R}, \\
    &u_0(\mathbf{x}): \overline{\Omega} \longrightarrow \mathbb{R}.\\
    \text{find} \quad &u(\mathbf{x},t) \in \overline{\Omega} \times \mathbb{T} \ \text{such that:}\\
    &\begin{cases}
        \begin{aligned}
            \frac{\partial u(\mathbf{x},t)}{\partial t} &- \frac{\partial^2u(\mathbf{x},t)}{\partial x^2}-\frac{\partial^2u(\mathbf{x},t)}{\partial^2y}=f(\mathbf{x},t) & \text{in} \quad &\Omega \times \mathbb{T} \\
            
            u(\mathbf{x},t) &=d_1(\mathbf{x},t) & \text{on} \quad &\Gamma_{D,1} \times \mathbb{T} \\
            
            u(\mathbf{x},t) &=d_2(\mathbf{x},t) & \text{on} \quad &\Gamma_{D,2} \times \mathbb{T} \\
            
            \frac{\partial u(\mathbf{x},t)}{\partial \mathbf{n}} &=1(\mathbf{x},t) & \text{on} \quad &\Gamma_{N} \times \mathbb{T} \\
            
            u(\mathbf{x},0) &= u_0(\mathbf{x}) & \text{in} \quad &\overline{\Omega} 
        \end{aligned}
    \end{cases}
\end{aligned}
\end{equation}
The problem is Homogeneous that means $f(\mathbf{x},t)=0$. The Dirichlet boundary condition for $\Gamma_D=\Gamma_{D,1}\cup \Gamma_{D,2}$ imposed $d_1(\mathbf{x},t)=0$ and $d_2(\mathbf{x},t)$ linear up to $1$, reached at $t=t_max/2$, then equal to $1$. The Neumann boundary condition on $\Gamma_N$ is given by the function $q(\mathbf{x},t)=0$ and the initial solution $u_0(\mathbf{x},t)=0$. The final time is $t_max=10$.
We have to solve the initial boundary value problem \eqref{eq:Heat_equation} with FEM using triangular elements and linear basis functions. 

\begin{figure}[h]
    \centering
    \includegraphics[width=0.5\textwidth]{Images/problem/domain.png}
        \caption{Domain $\Omega$ with boundary and points of interest}
        \label{fig: domain}
\end{figure}


